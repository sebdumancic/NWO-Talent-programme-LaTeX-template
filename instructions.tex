%\newcommand{\instructions}[2]{
%	\ifthenelse{\boolean{showinstructions}}{%
%		\begin{mdframed}
%    		\mdheader{#1}
%    		#2
%		\end{mdframed}
%	}{%
%    % Do nothing when flag is false
%  }%
%}


\newcommand{\guidelines}{%
	\instructions{Application form guidelines and important information}{%
	\textcolor{nwotext}{Formal requirements of the application form:}
	\begin{itemize}
		\item use the Calibri font (or similar), black, 10 point font size, except for references to the literature, which may be given in 9 point. Use line spacing 1 and do not change the margins (2,5 cm in either direction);
		\item texts that need editing, are optional or require a choice are marked in green and additionally marked with <<< at the beginning and >>> at the end of the text. Make sure that the markings and the text between them (and thus the green colour) are removed throughout the document before submitting;
		\item complete the application form entirely in English;
		\item limit yourself to the stated maximum number of words or pages for each section of the application form. Word counts include all text (including, but not limited to, references, footnotes, text in tables and figures). Exceptions are explicitly mentioned;
		\item please provide only the requested information. When asked for personal details mention initials and last name, and refrain from mentioning first name(s), to reduce gender effects; 
		\item it is not allowed to include letters of support or documents other than required;
		\item it is not allowed to include hyperlinks to a personal website, to a group website or to similar information which, in fact, extends the page limit set. The reviewers and committee members are only required to assess the information given in the online application system and appendices including this application form. In section 3, you may use an active hyperlink, on the condition that it links directly to the item. This hyperlink should preferably be in the form of a persistent identifier (e.g. a DOI).
		\item please submit the application form in PDF format free of security locks and bookmarks. If you do not know how to convert your application form to PDF format, allow extra time to get help from your institution’s computer support department or from the ISAAC helpdesk (see below);
		\item please remove the explanatory notes from the form, but leave all other instructions in the form. \\
	\end{itemize} 
	
	\textcolor{nwotext}{Important information, read before filling in the application form:}
	
	Please carefully read the information provided in the Explanatory Notes and the Call for proposals. 
	You can download the Call for proposals from the NWO-website (\href{https://www.nwo.nl/en/calls/nwo-talent-programme}{https://www.nwo.nl/en/calls/nwo-talent-programme}) or from ISAAC. 
	The original Dutch-language text of the Call for proposals is the authoritative version. 
	Where the English-language text is open to a different interpretation, no additional rights may be derived from it.
	
	Please make sure that all fields within your ISAAC-profile are up-to-date, such as your contact details (e.g. phone number, postal address). 
	This information will be used for administrative purposes. 
	Furthermore, please register your gender in ISAAC as this will influence policy decisions.
	
	
	To fill out the application form, you are free to use programmes other than Word (e.g., LaTeX), as long as you preserve the form’s overall structure and layout.
	
	The deadline for submitting your application is \textcolor{nwotext}{9 April 2026, 14:00:00 hrs CET}. 
	This means that you must have filled out all the information fields, uploaded all the documents, and have clicked the ‘submit’ button before 14:00:00 CET. 
	Applications received after the deadline are automatically disqualified.
	
	If you have any questions about the application form or application process, please do not hesitate to contact the programme coordinator of your domain. 
	Contact details can be found at the NWO-website (\href{https://www.nwo.nl/en/researchprogrammes/nwo-talent-programme}{https://www.nwo.nl/en/researchprogrammes/nwo-talent-programme}).
	
	Please bear in mind that within two weeks after the submission deadline, NWO may approach you with any possible administrative corrections that need to be made so that your application meets the conditions for submission. 
	You will receive a confirmation of the eligibility of your submission – i.e., whether it complies with all formal requirements – within approximately two to three weeks after the submission deadline.
	

	For any technical questions regarding submission, please contact the ISAAC helpdesk: 
	
	isaac.helpdesk@nwo.nl or +31 70 344 06 00.

	}
}




\newcommand{\generalinfo}{%
\instructions{}{%
\textcolor{nwotext}{\large Notes 1a. Title of research proposal} \\
Please ensure that the title provided at section 1a and entered in ISAAC are the same. \\


\textcolor{nwotext}{\large Notes 1b. Scientific summary of research proposal} \\ 
Provide a summary of your proposal (topic, approach and potential importance of the results) in no more than 300 words.

Make sure to provide an informative and relevant abstract, clearly describing what you are going to investigate, why you are going to investigate this subject and which results you expect to find. 
Please ensure that the summary provided in section 1b and entered in ISAAC are the same.


\textcolor{nwotext}{\large Notes 1c. Public summary of research proposal} \\ 
Please draft two public summaries of your proposal: one in Dutch and one in English. 
Also include a Dutch and an English popular title. 
Each public summary should be 50-100 words (excluding the title).


Please note that the public summaries are different from the scientific summary you have drafted under section 1b. 
Please ensure that the public summaries provided in section 1e and entered in ISAAC are the same.

Please keep the following guidelines in mind:
\begin{itemize}
	\item Use comprehensible, everyday language and be as specific as possible. For example, do not write “The mechanism underlying apoptosis will be examined” but “The researchers will use microscopes to look for the reasons for spontaneous cell death”.
	\item Do not write in terms of ‘we’ and ‘us’ but use terms like researchers, biologists, literary specialists, etc.
	\item Write the summary in such a way that you feel you ought to be including terms like ‘basically’, ‘put simply’, ‘roughly speaking’ and ‘in lay terms’ – but do not actually include them!
	\item We recommend that you have your public summaries proofread by a native speaker.
\end{itemize}

Examples of public summaries can be found at the NWO-website:
\begin{itemize}
	\item \href{https://www.nwo.nl/en/researchprogrammes/nwo-talent-programme/projects-vidi}{Project VIDI | NWO}
	\item \href{https://www.nwo.nl/onderzoeksprogrammas/nwo-talentprogramma/projecten-vidi}{Projecten VIDI | NWO}
\end{itemize}

If your application is successful, the public summaries will be used in NWO publicity surrounding the announcement of the grant award decisions.


\textcolor{nwotext}{\large Notes 1d. Main field of research} \\ 
Fill out one or more research fields that correspond to the subject of your research proposal.

Indicate the main field of research and (if applicable) other field(s) of research, in order of relevance, using the codes and names from the dropdown menu.

For the field(s) of research, you can refer to all listed research fields on the NWO research field list, using the exact codes and names (https://www.nwo.nl/en/nwo-research-fields). 
You can find the codes and research fields using the drop-down menu.

Please ensure that the field(s) of research provided in section 1d and entered in ISAAC are the same (on the tab “General Information” (Algemeen) section “Research fields” (Disciplines).
 In ISAAC, it is not possible to indicate a ‘Main field of research’, so just listing the chosen field(s) of research in ISAAC suffices.
 
Please note that in ISAAC, you only add the research field(s) and do not need to add the associated code manually. 
For example, if your main field of research is Business Administration you fill out the following:
\begin{itemize}
	\item In the application form: 39.90.00, Business Administration;
	\item In ISAAC: Business Administration.
\end{itemize}

Please note: ISAAC will list the research fields in the language of correspondence you have previously indicated in ISAAC (English or Dutch). 
You must search for your research field(s) in the language you selected (e.g., Business Administration or Bedrijfskunde).


\textcolor{nwotext}{\large Notes 1f. Prospective host institution} \\
List the institution that has provided the embedding guarantee. Add the department and the specific group (if applicable) at which you plan to execute your project. 
Please be as specific as possible and include the research group or lab. 
If the institution is changed since the pre-proposal, make sure to upload a new embedding guarantee.
}
}


\newcommand{\literaturenotes}{%
\instructions{Notes 3. Literature references}{%
The list of references should be relevant to the research proposal, cited in the texts of sections 2a and 2b. 
Be consistent in using the same reference format throughout the proposal and only include 'open literature'. 
It should not include internal documents (such as master theses) nor should it be a mere list of publications of the main applicant. 

Please note that it is not allowed to highlight publications with your own contribution in any way. 
To minimize potential gender effects, we kindly request to use a reference style that does not contain first names. 
In section 3, you may use an active hyperlink, on the condition that it links directly to the output item. 
If possible, you are strongly encouraged to include a DOI.


\textbf{This section does not count toward the word and/or page limitations.}

}
}


\newcommand{\researchinstructions}[1]{%
	\instructions{Research description notes}{%
		Applicants should be transparent about the use of generative AI (GAI) when writing the application. 
		If GAI is used as a source, this should be added into section 3 of the application. 
		The summarised GAI policy with guidelines for applicants, reviewers and NWO staff, as well as the full policy document, can be found on our website: 
		
		\href{https://www.nwo.nl/en/nwo-policy-on-the-use-of-generative-artificial-intelligence-gai}{NWO policy on the use of generative artificial intelligence (GAI) | NWO} 
		
		\textcolor{nwotext}{\large Notes 2a. Description of proposed research (Max 4000 Words)}
		
		Describe the proposed research as accurately as you can within the stated maximum amount of words and maximum number of pages.
		
		This maximum is 4000 words on no more than 8 pages. The word count includes all text used in section 2a (2a1, 2a2, 2a3 and 2a4), including in-text references, footnotes, figure captions, text in figures and tables.
		
		Include a description of the overall aim and key objectives (2a1), the research plan (2a2), the alignment between your expertise and the research proposal (2a3) and provide a concise motivation for choice of host institute (2a4).
		Take in mind that criterion 1 ‘Quality and innovative character of the research proposal’ and the underlying aspects will be assessed based on the information given in section 2a. 
		Please consult the Call for proposals for the full description of the criterion.

		
		These sections must at least contain:
		
		\textcolor{nwotext}{2a1. Overall aim and key objectives, including:}
		\begin{itemize}
			\item scientific relevance and challenges;
			\item originality and innovative character;
			\item methods and techniques.
		\end{itemize}
		
		
		\textcolor{nwotext}{2a2. Research plan, including:}
		\begin{itemize}
			\item practical timetable/timeline over the grant period, including information on the time spent by personnel;
			\item local, national and international collaboration;
			\item work plan (in narrative form).
		\end{itemize}
		
		\textcolor{nwotext}{2a3. Alignment between research proposal and expertise of the applicant:}
		\begin{itemize}
			\item description of the alignment between proposed research and the existing expertise of the applicant;
			\item if applicable: how this alignment can be achieved.
			\item 
		\end{itemize}
	
		The relevant expertise of the applicant should be described, but note that the CV itself is not part of the assessment criteria. 
		The information in the paragraph can be used in the assessment of ‘whether the proposal is in line with the researcher's expertise, or whether the researcher presents a convincing vision on how this alignment will be achieved’. 
		Keep in mind that the committee (and reviewers) do not have access to your preproposal.
	
		Do not include hyperlinks to personal websites or group websites. 
		Make sure the section complies with the DORA guidelines as described in Section 4.1 of the Call for proposals.
	
		Suggestion: keep the description concise (no more than a quarter of a page).

		\textcolor{nwotext}{2a4. Motivation for choice of host institute}
		
		Indicate why you prefer to carry out your research at the host institute, including:
		\begin{itemize}
			\item reasons for choosing this institute;
			\item in what way your research complements the already existing research line(s) at the host institute.
		\end{itemize}
		
		
		
		\textcolor{nwotext}{\large Notes 2b. Scientific and/or societal impact of proposed research (Max 1000 Words)}
		
		The scientific and/or societal impact of your proposed research should be described in maximum 1000 words on no more than 2 pages. 
		The word count includes all text used in section 2b including in-text references, footnotes, figure captions, text in figures and tables.
		
		
		You may choose to focus on achieving scientific impact, societal impact, or a combination of both. 
		Specify via the dropdown menu which kind of impact the proposal focuses on:
		\begin{itemize}
			\item Primary focus on scientific impact;
			\item Scientific and societal impact are of comparable focus;
			\item Primary focus on societal impact.
		\end{itemize}

		NWO assesses scientific impact as follows: whether the proposal conveys an ambitious vision and appropriate strategy regarding the dissemination and/or implementation of the research results in one’s own discipline, related disciplines and the broader scientific field.
		
		
		Societal impact refers to the societal (cultural, economic, industrial, ecological, or social) changes that result, in whole or in part, from knowledge and expertise generated through research. 
		These changes contribute to the well-being of people, the planet, and society, both now and for future generations.  
		
		
		It is not necessary to include both types of impact to get a good score for the criterion ‘Scientific and/or societal impact’. 
		A focus on scientific impact, a focus on societal impact, or a combination thereof, can all lead to a good score. 
		It is important that a clear motivation is given in the proposal for the choice to focus on scientific and/or societal impact. 
		Although you have the choice to focus on scientific and/or societal impact, you should consider how both foreseen and unforeseen opportunities can be seized for the non-chosen form of impact. 
		This may mean that you indicate in the proposal that (un)foreseen opportunities with regard to impact are being taken up by others, and that you describe what your own role is in making this possible.
		
		
		In order to increase the potential contribution of the project to the desired impact you are also required to describe the key risks of undesirable societal impact and proposed measures to prevent or mitigate these risks.
		
		
		On the basis of the referee reports and the rebuttal, the assessment committee will assess whether there are possibilities for impact that are not described in the proposal, and take this into account in its assessment.
		
		
		For further information on this criterion, please see the call for proposals and our \href{https://www.nwo.nl/en/knowledge-utilisation}{website}.


	}
}


\newcommand{\budgetinstructions}{
	\instructions{}{%
		\textcolor{nwotext}{Notes 4a. Budget summary}
		
		The budget should be filled in in our budget-Excel, which is available on our website and in ISAAC/Mijn ZonMw. 
		In the application form we ask you to include a summary of the budget-Excel. You should upload the budget-Excel as an attachment in ISAAC. 

		Per project, a grant amount of at most € 850,000 can be applied for. 
		Costs that can be applied for are specified in four categories: 1) Personnel costs, 2) Material costs, 3) Investments and 4) Knowledge utilisation.

		The maximum duration of the proposed project is five years. 
		If the proposed research is to be of shorter duration, the maximum amount of funding will be reduced accordingly. 
		If you wish to conduct the research on a part-time basis, the project duration cannot be extended, but it is possible to appoint additional scientific personnel. 
		However, this will have no effect on the maximum amount of the Vidi-grant (850,000 euros).

		Apply only for funding that is vital to realise the project. 
		The rates and an explanation of these budget modules are given in the Call for proposals, annex 7 and/or by the financial department of your host institution.

		What if the budget required to carry out the project exceeds the maximum amount of the Vidi-grant?
		It is possible that the total budget for your proposed research exceeds 850,000 euros. H
		owever, the contribution from NWO is fixed to maximum 850,000 euros. 
		This means that in case of a higher project budget, a contribution by the host institution and/or co-funding by a third party is required.

		For more information, please read section 3.5.6 Additional project contribution(s) of the Call for proposals.

		When filling out the budget table be aware of the following:
		\begin{itemize}
			\item costs for Open Access publishing should be included under ‘Materials/Implementation costs’;
		\end{itemize}


		\textcolor{nwotext}{\large Notes 4b. Budget justification}
		
		In section 4b you can explain in broad terms what resources will be required to conduct the proposed research. 
		You can include your motivation for your own contribution (in relation to the estimated FTE) and your choice for possibly (non-)scientific staff. 

		Also discuss and motivate your choices for budget decisions. 
		Provide a justification for the budget for knowledge utilisation, internationalisation, and a motivation for an excess in materials budget.
		
		\textcolor{nwotext}{\large Notes 4c. Interest/benefits for stakeholders (only AES)}
		
		This question is mandatory for AES applications, not applicable for SSH/Science. So, if applicable, mention the stakeholders that will be involved in the user committee of your proposed project. 
		If not applicable, you can indicate using the dropdown menu that this part is not applicable for your proposal. 

		This is mandatory for applications submitted within the AES domain, but not applicable to SSH and Science. 
		Applications submitted to NWO domain AES should include \textbf{a minimum of four users, of which at least two non-academic users}. 

		For every involved third party, please mention the name and/or logo of the organisation and motivate briefly why this party is involved: what are the benefits and interests for this party to participate in your proposed Vidi-project? Please do not use more than 70 words for the description of each user.
		
		\textcolor{nwotext}{Notes 4d. Additional (applications for) funding for overlapping project(s)}
		
		Please include details of any additional (application(s) for) funding for projects that partially or fully overlap with this application, whether from NWO or from any other institution (e.g. European Research Council). 

		Please note: Double funding is not permitted. If a (sub)project is funded by another (European) funding body, the entire Vidi application will not be awarded.
		
		It is not possible for NWO to fund costs that are funded through other sources, other than through the additional funding stated in the budget as contribution by the host institution (section 3b) and/or co-funding by a third party (section 3c).

		The explanation may be given in maximum 100 words.
	}
}


\newcommand{\datamangementinstr}{
	\instructions{Notes 5. Data management}{%
	\textcolor{nwotext}{Please answer the relevant questions with regard to the data management of your proposed research.}
	
	Responsible data management is part of good research. To promote effective and efficient data management, data sharing and data reuse, NWO expects researchers to carefully manage data resulting from NWO-funded research and prospectively plan for which data will be preserved and shared. With the data management section, NWO mainly wants to raise awareness about the importance of responsible data management. The section is therefore not included in a committee's decision about whether or not a proposal should be awarded funding. NWO does, however, submit this section to the external referees and the assessment committee for advice.
	
	It is recommended that you seek advice from a data steward or research support office at your host institution to complete this section. 
	They will be able to recommend suitable storage facilities and repositories for your data, and to advise on data management costs.
	
	After a proposal has been awarded funding, grantees are required to elaborate the data management section into a detailed data management plan explaining how research data and other results emerging from the NWO-funded research will be stored and made findable, accessible, interoperable and reusable (FAIR).
	
	
	\textcolor{nwotext}{What does NWO understand as research data?}
	
	Research data are the evidence that underpin the answer to research questions, and can be used to validate findings. Data can be quantitative information or qualitative statements collected by researchers in the course of their work by experimentation, observation, modelling, interview or other methods, or information derived from existing evidence.
	
	
	For the purpose of NWO’s data management policy, the definition of research data does not include physical objects such as scientific and archaeological collections, physical arts works or biobanks; however, digital information extracted from such objects are to be regarded as research data.
	
	Software is also not included in the definition of research data. NWO recognizes that software (algorithms, scripts and code developed by researchers in the course of their work) may be necessary to access and interpret data. In such cases, the data management plan will be expected to address how information about such items will be made available.
	
	
	\textcolor{nwotext}{What data does NWO expect you share and preserve?}
	
	Research results should be stored in such a way that they can be retrieved and reused in the long term, also by researchers in disciplines and organisations other than those in which the research took place. The operating principle is that all stored data are, in principle, freely accessible and that access is only limited if needed for reasons such as privacy, public security, ethical restrictions, property rights and commercial interests.

	NWO expects researchers to preserve the data resulting from their projects for at least ten years, unless legal provisions or discipline-specific guidelines dictate otherwise. As much as possible, research data should be made publicly available for reuse, unless there are valid reasons not to do so. As a minimum, NWO requires that the data underpinning research papers should be made available at the time of the article’s publication. Any tools or software (algorithms, scripts and code developed by researchers in the course of their work) necessary to access and interpret data should be made available alongside the data.
	
	The costs of data management are eligible for funding and can be included in the project budget.
	Important factors that determine the costs are:
	\begin{itemize}
		\item the type of data;
		\item the capacity needed for storage and backup;
		\item the amount of work needed to allocate metadata and the compilation of other documentation such as codebooks and the queries used in the statistical package;
		\item the extent to which the data needs to be protected;
		\item the hiring in of external data management expertise or other expertise.
	\end{itemize}
	
	}
}